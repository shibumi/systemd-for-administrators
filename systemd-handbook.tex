\documentclass[titlepage]{article}
\title{Systemd for Administrators}
\usepackage[english]{babel}
\usepackage{hyperref}
\usepackage{listings}
\author{Lennart Poettering}
\lstset{
basicstyle=\tiny,
frame=single
}
\begin{document}
\maketitle
\tableofcontents
\newpage

\section{Abstract}
As many of you know,
\href{https://www.freedesktop.org/wiki/Software/systemd}{systemd} is the new
Fedora init system, starting with F14, and it is also on its way to being
adopted in a number of other distributions as well (for example,
\href{https://en.opensuse.org/SDB:Systemd}{OpenSUSE}). For administrators
systemd provides a variety of new features and changes and enhances the
administrative process substantially. This blog story is the first part of a
series of articles I plan to post roughly every week for the next months. In
every post I will try to explain one new feature of systemd. Many of these
features are small and simple, so these stories should be interesting to a
broader audience. However, from time to time we'll dive a little bit deeper
into the great new features systemd provides you with.  
\newpage
\section{Verifying Bootup}
Traditionally, when booting up a Linux system, you see a lot of little
messages passing by on your screen. As we work on speeding up and
parallelizing the boot process these messages are becoming visible for a
shorter and shorter time only and be less and less readable -- if they are
shown at all, given we use graphical boot splash technology like Plymouth
these days. Nonetheless the information of the boot screens was and still is
very relevant, because it shows you for each service that is being started
as part of bootup, wether it managed to start up successfully or failed
(with those green or red [ OK ] or [ FAILED ] indicators). To improve the
situation for machines that boot up fast and parallelized and to make this
information more nicely available during runtime, we added a feature to
systemd that tracks and remembers for each service whether it started up
successfully, whether it exited with a non-zero exit code, whether it timed
out, or whether it terminated abnormally (by segfaulting or similar), both
during start-up and runtime. By simply typing systemctl in your shell you
can query the state of all services, both systemd native and SysV/LSB
services:
\begin{lstlisting}
[root@lambda] ~# systemctl
UNIT                                          LOAD   ACTIVE       SUB        
dev-hugepages.automount                       loaded active       running    
dev-mqueue.automount                          loaded active       running    
proc-sys-fs-binfmt_misc.automount             loaded active       waiting    
sys-kernel-debug.automount                    loaded active       waiting    
sys-kernel-security.automount                 loaded active       waiting    
sys-devices-pc...0000:02:00.0-net-eth0.device loaded active       plugged    
sys-devices-virtual-tty-tty9.device           loaded active       plugged    
-.mount                                       loaded active       mounted    
boot.mount                                    loaded active       mounted    
dev-hugepages.mount                           loaded active       mounted    
dev-mqueue.mount                              loaded active       mounted    
home.mount                                    loaded active       mounted    
proc-sys-fs-binfmt_misc.mount                 loaded active       mounted    
abrtd.service                                 loaded active       running    
bus.service                                  loaded active       running    
getty@tty2.service                            loaded active       running    
getty@tty3.service                            loaded active       running    
getty@tty4.service                            loaded active       running    
getty@tty5.service                            loaded active       running    
getty@tty6.service                            loaded active       running    
haldaemon.service                             loaded active       running    
hdapsd@sda.service                            loaded active       running    
irqbalance.service                            loaded active       running    
iscsi.service                                 loaded active       exited     
iscsid.service                                loaded active       exited     
livesys-late.service                          loaded active       exited     
livesys.service                               loaded active       exited     
lvm2-monitor.service                          loaded active       exited     
mdmonitor.service                             loaded active       running    
modem-manager.service                         loaded active       running    
netfs.service                                 loaded active       exited     
NetworkManager.service                        loaded active       running    
ntpd.service                                  loaded maintenance  maintenance

LOAD   = Reflects whether the unit definition was properly loaded.
ACTIVE = The high-level unit activation state, i.e. generalization of SUB.
SUB    = The low-level unit activation state, values depend on unit type.
JOB    = Pending job for the unit.

221 units listed. Pass --all to see inactive units, too.
[root@lambda] ~#
\end{lstlisting}
\section{References}
\nocite{*}
\bibliography{references}
\bibliographystyle{plain}
\end{document}
