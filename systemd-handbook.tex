\documentclass[titlepage]{article}
\title{Systemd for Administrators}
\usepackage[english]{babel}
\usepackage{hyperref}
\usepackage{listings}
\author{
    Lennart Poettering\\
    \texttt{Author}
    \and
    Christian Rebischke\\
    \texttt{Editor}
}
\lstset{
    basicstyle=\tiny,
    frame=single
}
\setlength{\parindent}{0pt}
\begin{document}
\maketitle
\tableofcontents
\newpage

\section{Abstract}
As many of you know,
\href{https://www.freedesktop.org/wiki/Software/systemd}{systemd} is the new
Fedora init system, starting with F14, and it is also on its way to being
adopted in a number of other distributions as well (for example,
\href{https://en.opensuse.org/SDB:Systemd}{OpenSUSE}). For administrators
systemd provides a variety of new features and changes and enhances the
administrative process substantially. This blog story is the first part of a
series of articles I plan to post roughly every week for the next months. In
every post I will try to explain one new feature of systemd. Many of these
features are small and simple, so these stories should be interesting to a
broader audience. However, from time to time we'll dive a little bit deeper
into the great new features systemd provides you with.  
\section{Disclaimer}
This handbook is written by Lennart Poettering. There are maybe some 
additions, cuts or other changes for increasing readability.
Please visit Lennarts Blog for the original Blogposts:
\\
\\
\url{https://0pointer.de/blog}
\newpage
\section{Verifying Bootup}
Traditionally, when booting up a Linux system, you see a lot of little
messages passing by on your screen. As we work on speeding up and
parallelizing the boot process these messages are becoming visible for a
shorter and shorter time only and be less and less readable -- if they are
shown at all, given we use graphical boot splash technology like Plymouth
these days. Nonetheless the information of the boot screens was and still is
very relevant, because it shows you for each service that is being started
as part of bootup, wether it managed to start up successfully or failed
(with those green or red [ OK ] or [ FAILED ] indicators). To improve the
situation for machines that boot up fast and parallelized and to make this
information more nicely available during runtime, we added a feature to
systemd that tracks and remembers for each service whether it started up
successfully, whether it exited with a non-zero exit code, whether it timed
out, or whether it terminated abnormally (by segfaulting or similar), both
during start-up and runtime. By simply typing systemctl in your shell you
can query the state of all services, both systemd native and SysV/LSB
services:
\begin{lstlisting}
[root@lambda] ~# systemctl
UNIT                                          LOAD   ACTIVE       SUB        
dev-hugepages.automount                       loaded active       running    
dev-mqueue.automount                          loaded active       running    
proc-sys-fs-binfmt_misc.automount             loaded active       waiting    
sys-kernel-debug.automount                    loaded active       waiting    
sys-kernel-security.automount                 loaded active       waiting    
sys-devices-pc...0000:02:00.0-net-eth0.device loaded active       plugged    
sys-devices-virtual-tty-tty9.device           loaded active       plugged    
-.mount                                       loaded active       mounted    
boot.mount                                    loaded active       mounted    
dev-hugepages.mount                           loaded active       mounted    
dev-mqueue.mount                              loaded active       mounted    
home.mount                                    loaded active       mounted    
proc-sys-fs-binfmt_misc.mount                 loaded active       mounted    
abrtd.service                                 loaded active       running    
bus.service                                  loaded active       running    
getty@tty2.service                            loaded active       running    
getty@tty3.service                            loaded active       running    
getty@tty4.service                            loaded active       running    
getty@tty5.service                            loaded active       running    
getty@tty6.service                            loaded active       running    
haldaemon.service                             loaded active       running    
hdapsd@sda.service                            loaded active       running    
irqbalance.service                            loaded active       running    
iscsi.service                                 loaded active       exited     
iscsid.service                                loaded active       exited     
livesys-late.service                          loaded active       exited     
livesys.service                               loaded active       exited     
lvm2-monitor.service                          loaded active       exited     
mdmonitor.service                             loaded active       running    
modem-manager.service                         loaded active       running    
netfs.service                                 loaded active       exited     
NetworkManager.service                        loaded active       running    
ntpd.service                                  loaded maintenance  maintenance

LOAD   = Reflects whether the unit definition was properly loaded.
ACTIVE = The high-level unit activation state, i.e. generalization of SUB.
SUB    = The low-level unit activation state, values depend on unit type.
JOB    = Pending job for the unit.

221 units listed. Pass --all to see inactive units, too.
[root@lambda] ~#
\end{lstlisting}
(I have shortened the output above a little, and removed a few lines not relevant for this blog post.)
Look at the ACTIVE column, which shows you the high-level state of a service (or in fact of any kind of unit systemd maintains, which can be more than just services, but we'll have a look on this in a later blog posting), whether it is \textbf{active} (i.e. running), \textbf{inactive} (i.e. not running) or in any other state. If you look closely you'll see one item in the list that is marked \textbf{maintenance} and highlighted in red. This informs you about a service that failed to run or otherwise encountered a problem. In this case this is ntpd. Now, let's find out what actually happened to ntpd, with the \textit{systemctl status} command:
\begin{lstlisting}
[root@lambda] ~# systemctl status ntpd.service
ntpd.service - Network Time Service
      Loaded: loaded (/etc/systemd/system/ntpd.service)
      Active: maintenance
        Main: 953 (code=exited, status=255)
      CGroup: name=systemd:/systemd-1/ntpd.service
[root@lambda] ~#
\end{lstlisting}
This shows us that NTP terminated during runtime (when it ran as PID 953), and tells us exactly the error condition: the process exited with an exit status of 255.
\\
\\
In a later systemd version, we plan to hook this up to ABRT, \href{https://bugzilla.redhat.com/show_bug.cgi?id=622773}{as soon as this enhancement request is fixed}. Then, if systemctl status shows you information about a service that crashed it will direct you right-away to the appropriate crash dump in ABRT.
\\
\\
\textbf{Summary}: use \textit{systemctl} and \textit{systemctl status} as modern, more complete replacements for the traditional boot-up status messages of SysV services. \textit{systemctl status} not only captures in more detail the error condition but also shows runtime errors in addition to start-up errors.
That's it for this week, make sure to come back next week, for the next posting about systemd for administrators!
\section{Which Service Owns Which Processes?}
On most Linux systems the number of processes that are running by default is substantial. Knowing which process does what and where it belongs to becomes increasingly difficult. Some services even maintain a couple of worker processes which clutter the "ps" output with many additional processes that are often not easy to recognize. This is further complicated if daemons spawn arbitrary 3rd-party processes, as Apache does with CGI processes, or cron does with user jobs.
\\
\\
A slight remedy for this is often the process inheritance tree, as shown by "ps xaf". However this is usually not reliable, as processes whose parents die get reparented to PID 1, and hence all information about inheritance gets lost. If a process "double forks" it hence loses its relationships to the processes that started it. (This actually is supposed to be a feature and is relied on for the traditional Unix daemonizing logic.) Furthermore processes can freely change their names with PR\_SETNAME or by patching argv[0], thus making it harder to recognize them. In fact they can play hide-and-seek with the administrator pretty nicely this way.
\\
\\
In systemd we place every process that is spawned in a control group named after its service. Control groups (or cgroups) at their most basic are simply groups of processes that can be arranged in a hierarchy and labelled individually. When processes spawn other processes these children are automatically made members of the parents cgroup. Leaving a cgroup is not possible for unprivileged processes. Thus, cgroups can be used as an effective way to label processes after the service they belong to and be sure that the service cannot escape from the label, regardless how often it forks or renames itself. Furthermore this can be used to safely kill a service and all processes it created, again with no chance of escaping.
\\
\\
In today's installment I want to introduce you to two commands you may use to relate systemd services and processes. The first one, is the well known ps command which has been updated to show cgroup information along the other process details. And this is how it looks:
\begin{lstlisting}
$ ps xawf -eo pid,user,cgroup,args
  PID USER     CGROUP                              COMMAND
    2 root     -                                   [kthreadd]
    3 root     -                                    \_ [ksoftirqd/0]
[...]
 4281 root     -                                    \_ [flush-8:0]
    1 root     name=systemd:/systemd-1             /sbin/init
  455 root     name=systemd:/systemd-1/sysinit.service /sbin/udevd -d
28188 root     name=systemd:/systemd-1/sysinit.service  \_ /sbin/udevd -d
28191 root     name=systemd:/systemd-1/sysinit.service  \_ /sbin/udevd -d
 1131 root     name=systemd:/systemd-1/auditd.service auditd
 1133 root     name=systemd:/systemd-1/auditd.service  \_ /sbin/audispd
 1135 root     name=systemd:/systemd-1/auditd.service      \_ /usr/sbin/sedispatch
 1193 root     name=systemd:/systemd-1/rsyslog.service /sbin/rsyslogd -c 4
 1195 root     name=systemd:/systemd-1/cups.service cupsd -C /etc/cups/cupsd.conf
 1210 root     name=systemd:/systemd-1/irqbalance.service irqbalance
 1216 root     name=systemd:/systemd-1/dbus.service /usr/sbin/modem-manager
 1219 root     name=systemd:/systemd-1/dbus.service /usr/libexec/polkit-1/polkitd
1317 root     name=systemd:/systemd-1/abrtd.service /usr/sbin/abrtd -d -s
 1332 root     name=systemd:/systemd-1/getty@.service/tty2 /sbin/mingetty tty2
 1339 root     name=systemd:/systemd-1/getty@.service/tty3 /sbin/mingetty tty3
 1342 root     name=systemd:/systemd-1/getty@.service/tty5 /sbin/mingetty tty5
 1343 root     name=systemd:/systemd-1/getty@.service/tty4 /sbin/mingetty tty4
 1344 root     name=systemd:/systemd-1/crond.service crond
 1346 root     name=systemd:/systemd-1/getty@.service/tty6 /sbin/mingetty tty6
 1362 root     name=systemd:/systemd-1/sshd.service /usr/sbin/sshd
 1759 lennart  name=systemd:/user/lennart/1        gnome-screensaver
 909 lennart  name=systemd:/user/lennart/1        gnome-terminal
 1913 lennart  name=systemd:/user/lennart/1         \_ gnome-pty-helper
 1914 lennart  name=systemd:/user/lennart/1         \_ bash
29231 lennart  name=systemd:/user/lennart/1         |   \_ ssh tango
 2221 lennart  name=systemd:/user/lennart/1         \_ bash
 4193 lennart  name=systemd:/user/lennart/1         |   \_ ssh tango
 2461 lennart  name=systemd:/user/lennart/1         \_ bash
27251 lennart  name=systemd:/user/lennart/1             \_ empathy
\end{lstlisting}
(Note that this output is shortened, I have removed most of the kernel threads here, since they are not relevant in the context of this blog story)
\\
\\
In the third column you see the cgroup systemd assigned to each process. You'll find that the udev processes are in the name=systemd:/systemd-1/sysinit.service cgroup, which is where systemd places all processes started by the sysinit.service service, which covers early boot.
\\
\\
\newpage
My personal recommendation is to set the shell alias psc to the ps command line shown above:
\begin{lstlisting}
alias psc='ps xawf -eo pid,user,cgroup,args'
\end{lstlisting}
With this service information of processes is just four keypresses away! A different way to present the same information is the systemd-cgls tool we ship with systemd. It shows the cgroup hierarchy in a pretty tree. Its output looks like this:
\begin{lstlisting}
$ systemd-cgls
+    2 [kthreadd]
[...]
+ 4281 [flush-8:0]
+ user
| \ lennart
|   \ 1
|     +  1495 pam: gdm-password
|     +  1521 gnome-session
|     +  1534 dbus-launch --sh-syntax --exit-with-session
|     +  1603 /usr/libexec/gconfd-2
|     +  1612 /usr/libexec/gnome-settings-daemon
|     +  1615 /ushr/libexec/gvfsd
|     \ 29519 systemd-cgls
\ systemd-1
  + 1 /sbin/init
  + ntpd.service
  | \ 4112 /usr/sbin/ntpd -n -u ntp:ntp -g
  + systemd-logger.service
  | \ 1499 /lib/systemd/systemd-logger
  + accounts-daemon.service
  | \ 1496 /usr/libexec/accounts-daemon
  + rtkit-daemon.service
  | \ 1473 /usr/libexec/rtkit-daemon
  + console-kit-daemon.service
  | \ 1408 /usr/sbin/console-kit-daemon --no-daemon
  + prefdm.service
  | + 1376 /usr/sbin/gdm-binary -nodaemon
  | + 1419 /usr/bin/dbus-launch --exit-with-session
  | \ 1511 /usr/bin/gnome-keyring-daemon --daemonize --login
  + getty@.service
  | + tty6
  | | \ 1346 /sbin/mingetty tty6
  | + tty4
  | | \ 1343 /sbin/mingetty tty4
  | + tty5
  | | \ 1342 /sbin/mingetty tty5
  | + tty3
  | | \ 1339 /sbin/mingetty tty3
  | \ tty2
  |   \ 1332 /sbin/mingetty tty2
   \ 28191 /sbin/udevd -d
\end{lstlisting}
(This too is shortened, the same way)
\\
\\
As you can see, this command shows the processes by their cgroup and hence service, as systemd labels the cgroups after the services. For example, you can easily see that the auditing service auditd.service spawns three individual processes, auditd, audisp and sedispatch.
\\
\\
If you look closely you will notice that a number of processes have been assigned to the cgroup /user/1. At this point let's simply leave it at that systemd not only maintains services in cgroups, but user session processes as well. In a later installment we'll discuss in more detail what this about.
\\
\\
So much for now, come back soon for the next installment!
\section{How Do I Convert A SysV Init Script Into A systemd Service File?}
Traditionally, Unix and Linux services (daemons) are started via SysV init scripts. These are Bourne Shell scripts, usually residing in a directory such as /etc/rc.d/init.d/ which when called with one of a few standardized arguments (verbs) such as start, stop or restart controls, i.e. starts, stops or restarts the service in question. For starts this usually involves invoking the daemon binary, which then forks a background process (more precisely daemonizes). Shell scripts tend to be slow, needlessly hard to read, very verbose and fragile. Although they are immensly flexible (after all, they are just code) some things are very hard to do properly with shell scripts, such as ordering parallized execution, correctly supervising processes or just configuring execution contexts in all detail. systemd provides compatibility with these shell scripts, but due to the shortcomings pointed out it is recommended to install native systemd service files for all daemons installed. Also, in contrast to SysV init scripts which have to be adjusted to the distribution systemd service files are compatible with any kind of distribution running systemd (which become more and more these days...). What follows is a terse guide how to take a SysV init script and translate it into a native systemd service file. Ideally, upstream projects should ship and install systemd service files in their tarballs. If you have successfully converted a SysV script according to the guidelines it might hence be a good idea to submit the file as patch to upstream. How to prepare a patch like that will be discussed in a later installment, suffice to say at this point that the \href{https://0pointer.de/public/systemd-man/daemon.html}{daemon(7)} manual page shipping with systemd contains a lot of useful information regarding this.
\\
\\
So, let's jump right in. As an example we'll convert the init script of the ABRT daemon into a systemd service file. ABRT is a standard component of every Fedora install, and is an acronym for Automatic Bug Reporting Tool, which pretty much describes what it does, i.e. it is a service for collecting crash dumps. \href{https://0pointer.de/public/abrtd}{Its SysV script I have uploaded here}.
\newpage
The first step when converting such a script is to read it (surprise surprise!) and distill the useful information from the usually pretty long script. In almost all cases the script consists of mostly boilerplate code that is identical or at least very similar in all init scripts, and usually copied and pasted from one to the other. So, let's extract the interesting information from the script linked above:
\begin{itemize}
\item A description string for the service is "Daemon to detect crashing apps". As it turns out, the header comments include a redundant number of description strings, some of them describing less the actual service but the init script to start it. systemd services include a description too, and it should describe the service and not the service file.
\item The LSB header[1] contains dependency information. systemd due to its design around socket-based activation usually needs no (or very little) manually configured dependencies. (For details regarding socket activation \href{https://0pointer.de/blog/projects/systemd.html}{see the original announcement blog post}.) In this case the dependency on \$syslog (which encodes that abrtd requires a syslog daemon), is the only valuable information. While the header lists another dependency (\$local\_fs) this one is redundant with systemd as normal system services are always started with all local file systems available.
\item The LSB header suggests that this service should be started in runlevels 3 (multi-user) and 5 (graphical).
\item The daemon binary is /usr/sbin/abrtd
\end{itemize}
And that's already it. The entire remaining content of this 115-line shell script is simply boilerplate or otherwise redundant code: code that deals with synchronizing and serializing startup (i.e. the code regarding lock files) or that outputs status messages (i.e. the code calling echo), or simply parsing of the verbs (i.e. the big case block).
\\
\\
From the information extracted above we can now write our systemd service file:
\begin{lstlisting}
[Unit]
Description=Daemon to detect crashing apps
After=syslog.target

[Service]
ExecStart=/usr/sbin/abrtd
Type=forking

[Install]
WantedBy=multi-user.target
\end{lstlisting}
A little explanation of the contents of this file: The [Unit] section contains generic information about the service. systemd not only manages system services, but also devices, mount points, timer, and other components of the system. The generic term for all these objects in systemd is a unit, and the [Unit] section encodes information about it that might be applicable not only to services but also in to the other unit types systemd maintains. In this case we set the following unit settings: we set the description string and configure that the daemon shall be started after Syslog[2], similar to what is encoded in the LSB header of the original init script. For this Syslog dependency we create a dependency of type After= on a systemd unit syslog.target. The latter is a special target unit in systemd and is the standardized name to pull in a syslog implementation. For more information about these standardized names see the \href{https://0pointer.de/public/systemd-man/systemd.special.html}{systemd.special(7)}. Note that a dependency of type After= only encodes the suggested ordering, but does not actually cause syslog to be started when abrtd is -- and this is exactly what we want, since abrtd actually works fine even without syslog being around. However, if both are started (and usually they are) then the order in which they are is controlled with this dependency.
\\
\\
The next section is [Service] which encodes information about the service itself. It contains all those settings that apply only to services, and not the other kinds of units systemd maintains (mount points, devices, timers, ...). Two settings are used here: ExecStart= takes the path to the binary to execute when the service shall be started up. And with Type= we configure how the service notifies the init system that it finished starting up. Since traditional Unix daemons do this by returning to the parent process after having forked off and initialized the background daemon we set the type to forking here. That tells systemd to wait until the start-up binary returns and then consider the processes still running afterwards the daemon processes.
\\
\\
The final section is [Install]. It encodes information about how the suggested installation should look like, i.e. under which circumstances and by which triggers the service shall be started. In this case we simply say that this service shall be started when the multi-user.target unit is activated. This is a special unit (see above) that basically takes the role of the classic SysV Runlevel 3[3]. The setting WantedBy= has little effect on the daemon during runtime. It is only read by the systemctl enable command, which is the recommended way to enable a service in systemd. This command will simply ensure that our little service gets automatically activated as soon as multi-user.target is requested, which it is on all normal boots[4].
\\
\\
And that's it. Now we already have a minimal working systemd service file. To test it we copy it to /etc/systemd/system/abrtd.service and invoke systemctl daemon-reload. This will make systemd take notice of it, and now we can start the service with it: systemctl start abrtd.service. We can verify the status via systemctl status abrtd.service. And we can stop it again via systemctl stop abrtd.service. Finally, we can enable it, so that it is activated by default on future boots with systemctl enable abrtd.service.
\\
\\
The service file above, while sufficient and basically a 1:1 translation (feature- and otherwise) of the SysV init script still has room for improvement. Here it is a little bit updated:
\begin{lstlisting}
[Unit]
Description=ABRT Automated Bug Reporting Tool
After=syslog.target

[Service]
Type=dbus
BusName=com.redhat.abrt
ExecStart=/usr/sbin/abrtd -d -s

[Install]
WantedBy=multi-user.target
\end{lstlisting}
So, what did we change? Two things: we improved the description string a bit. More importantly however, we changed the type of the service to dbus and configured the D-Bus bus name of the service. Why did we do this? As mentioned classic SysV services daemonize after startup, which usually involves double forking and detaching from any terminal. While this is useful and necessary when daemons are invoked via a script, this is unnecessary (and slow) as well as counterproductive when a proper process babysitter such as systemd is used. The reason for that is that the forked off daemon process usually has little relation to the original process started by systemd (after all the daemonizing scheme's whole idea is to remove this relation), and hence it is difficult for systemd to figure out after the fork is finished which process belonging to the service is actually the main process and which processes might just be auxiliary. But that information is crucial to implement advanced babysitting, i.e. supervising the process, automatic respawning on abnormal termination, collectig crash and exit code information and suchlike. In order to make it easier for systemd to figure out the main process of the daemon we changed the service type to dbus. The semantics of this service type are appropriate for all services that take a name on the D-Bus system bus as last step of their initialization[5]. ABRT is one of those. With this setting systemd will spawn the ABRT process, which will no longer fork (this is configured via the -d -s switches to the daemon), and systemd will consider the service fully started up as soon as com.redhat.abrt appears on the bus. This way the process spawned by systemd is the main process of the daemon, systemd has a reliable way to figure out when the daemon is fully started up and systemd can easily supervise it.

And that's all there is to it. We have a simple systemd service file now that encodes in 10 lines more information than the original SysV init script encoded in 115. And even now there's a lot of room left for further improvement utilizing more features systemd offers. For example, we could set Restart=restart-always to tell systemd to automatically restart this service when it dies. Or, we could use OOMScoreAdjust=-500 to ask the kernel to please leave this process around when the OOM killer wreaks havoc. Or, we could use CPUSchedulingPolicy=idle to ensure that abrtd processes crash dumps in background only, always allowing the kernel to give preference to whatever else might be running and needing CPU time.
\\
\\
For more information about the configuration options mentioned here, see the respective man pages \href{https://0pointer.de/public/systemd-man/systemd.unit.html}{systemd.unit(5)}, \href{https://0pointer.de/public/systemd-man/systemd.service.html}{systemd.service(5)}, \href{https://0pointer.de/public/systemd-man/systemd.exec.html}{systemd.exec(5)}. Or, browse \href{https://0pointer.de/public/systemd-man/}{all of systemd's man pages}.
\\
\\
Of course, not all SysV scripts are as easy to convert as this one. But gladly, as it turns out the vast majority actually are.
\\
\\
That's it for today, come back soon for the next installment in our series.
\section{References}
\nocite{*}
\bibliography{references}
\bibliographystyle{plain}
\end{document}
