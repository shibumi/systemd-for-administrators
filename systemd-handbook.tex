\documentclass[titlepage]{article}
\title{Systemd for Administrators}
\usepackage[english]{babel}
\usepackage{hyperref}
\usepackage{listings}
\author{Lennart Poettering}
\lstset{
basicstyle=\footnotesize,
frame=single
}
\begin{document}
\maketitle
\tableofcontents
\newpage

\section{Abstract}
As many of you know,
\href{https://www.freedesktop.org/wiki/Software/systemd}{systemd} is the new
Fedora init system, starting with F14, and it is also on its way to being
adopted in a number of other distributions as well (for example,
\href{https://en.opensuse.org/SDB:Systemd}{OpenSUSE}). For administrators
systemd provides a variety of new features and changes and enhances the
administrative process substantially. This blog story is the first part of a
series of articles I plan to post roughly every week for the next months. In
every post I will try to explain one new feature of systemd. Many of these
features are small and simple, so these stories should be interesting to a
broader audience. However, from time to time we'll dive a little bit deeper
into the great new features systemd provides you with.  
\newpage
\section{Verifying Bootup}
Traditionally, when booting up a Linux system, you see a lot of little
messages passing by on your screen. As we work on speeding up and
parallelizing the boot process these messages are becoming visible for a
shorter and shorter time only and be less and less readable -- if they are
shown at all, given we use graphical boot splash technology like Plymouth
these days. Nonetheless the information of the boot screens was and still is
very relevant, because it shows you for each service that is being started
as part of bootup, wether it managed to start up successfully or failed
(with those green or red [ OK ] or [ FAILED ] indicators). To improve the
situation for machines that boot up fast and parallelized and to make this
information more nicely available during runtime, we added a feature to
systemd that tracks and remembers for each service whether it started up
successfully, whether it exited with a non-zero exit code, whether it timed
out, or whether it terminated abnormally (by segfaulting or similar), both
during start-up and runtime. By simply typing systemctl in your shell you
can query the state of all services, both systemd native and SysV/LSB
services:
\begin{lstlisting}
[root@lambda] ~# systemctl
UNIT                                          LOAD   ACTIVE       SUB          JOB             DESCRIPTION
dev-hugepages.automount                       loaded active       running                      Huge Pages File System Automount Point
dev-mqueue.automount                          loaded active       running                      POSIX Message Queue File System Automount Point
proc-sys-fs-binfmt_misc.automount             loaded active       waiting                      Arbitrary Executable File Formats File System Automount Point
sys-kernel-debug.automount                    loaded active       waiting                      Debug File System Automount Point
sys-kernel-security.automount                 loaded active       waiting                      Security File System Automount Point
sys-devices-pc...0000:02:00.0-net-eth0.device loaded active       plugged                      82573L Gigabit Ethernet Controller
[...]
sys-devices-virtual-tty-tty9.device           loaded active       plugged                      /sys/devices/virtual/tty/tty9
-.mount                                       loaded active       mounted                      /
boot.mount                                    loaded active       mounted                      /boot
dev-hugepages.mount                           loaded active       mounted                      Huge Pages File System
dev-mqueue.mount                              loaded active       mounted                      POSIX Message Queue File System
home.mount                                    loaded active       mounted                      /home
proc-sys-fs-binfmt_misc.mount                 loaded active       mounted                      Arbitrary Executable File Formats File System
abrtd.service                                 loaded active       running                      ABRT Automated Bug Reporting Tool
accounts-daemon.service                       loaded active       running                      Accounts Service
acpid.service                                 loaded active       running                      ACPI Event Daemon
atd.service                                   loaded active       running                      Execution Queue Daemon
auditd.service                                loaded active       running                      Security Auditing Service
avahi-daemon.service                          loaded active       running                      Avahi mDNS/DNS-SD Stack
bluetooth.service                             loaded active       running                      Bluetooth Manager
console-kit-daemon.service                    loaded active       running                      Console Manager
cpuspeed.service                              loaded active       exited                       LSB: processor frequency scaling support
crond.service                                 loaded active       running                      Command Scheduler
cups.service                                  loaded active       running                      CUPS Printing Service
dbus.service                                  loaded active       running                      D-Bus System Message Bus
getty@tty2.service                            loaded active       running                      Getty on tty2
getty@tty3.service                            loaded active       running                      Getty on tty3
getty@tty4.service                            loaded active       running                      Getty on tty4
getty@tty5.service                            loaded active       running                      Getty on tty5
getty@tty6.service                            loaded active       running                      Getty on tty6
haldaemon.service                             loaded active       running                      Hardware Manager
hdapsd@sda.service                            loaded active       running                      sda shock protection daemon
irqbalance.service                            loaded active       running                      LSB: start and stop irqbalance daemon
iscsi.service                                 loaded active       exited                       LSB: Starts and stops login and scanning of iSCSI devices.
iscsid.service                                loaded active       exited                       LSB: Starts and stops login iSCSI daemon.
livesys-late.service                          loaded active       exited                       LSB: Late init script for live image.
livesys.service                               loaded active       exited                       LSB: Init script for live image.
lvm2-monitor.service                          loaded active       exited                       LSB: Monitoring of LVM2 mirrors, snapshots etc. using dmeventd or progress polling
mdmonitor.service                             loaded active       running                      LSB: Start and stop the MD software RAID monitor
modem-manager.service                         loaded active       running                      Modem Manager
netfs.service                                 loaded active       exited                       LSB: Mount and unmount network filesystems.
NetworkManager.service                        loaded active       running                      Network Manager
ntpd.service                                  loaded maintenance  maintenance                  Network Time Service
polkitd.service                               loaded active       running                      Policy Manager
prefdm.service                                loaded active       running                      Display Manager
rc-local.service                              loaded active       exited                       /etc/rc.local Compatibility
rpcbind.service                               loaded active       running                      RPC Portmapper Service
rsyslog.service                               loaded active       running                      System Logging Service
rtkit-daemon.service                          loaded active       running                      RealtimeKit Scheduling Policy Service
sendmail.service                              loaded active       running                      LSB: start and stop sendmail
sshd@172.31.0.53:22-172.31.0.4:36368.service  loaded active       running                      SSH Per-Connection Server
sysinit.service                               loaded active       running                      System Initialization
systemd-logger.service                        loaded active       running                      systemd Logging Daemon
udev-post.service                             loaded active       exited                       LSB: Moves the generated persistent udev rules to /etc/udev/rules.d
udisks.service                                loaded active       running                      Disk Manager
upowerd.service                               loaded active       running                      Power Manager
wpa_supplicant.service                        loaded active       running                      Wi-Fi Security Service
avahi-daemon.socket                           loaded active       listening                    Avahi mDNS/DNS-SD Stack Activation Socket
cups.socket                                   loaded active       listening                    CUPS Printing Service Sockets
dbus.socket                                   loaded active       running                      dbus.socket
rpcbind.socket                                loaded active       listening                    RPC Portmapper Socket
sshd.socket                                   loaded active       listening                    sshd.socket
systemd-initctl.socket                        loaded active       listening                    systemd /dev/initctl Compatibility Socket
systemd-logger.socket                         loaded active       running                      systemd Logging Socket
systemd-shutdownd.socket                      loaded active       listening                    systemd Delayed Shutdown Socket
dev-disk-by\x1...x1db22a\x1d870f1adf2732.swap loaded active       active                       /dev/disk/by-uuid/fd626ef7-34a4-4958-b22a-870f1adf2732
basic.target                                  loaded active       active                       Basic System
bluetooth.target                              loaded active       active                       Bluetooth
dbus.target                                   loaded active       active                       D-Bus
getty.target                                  loaded active       active                       Login Prompts
graphical.target                              loaded active       active                       Graphical Interface
local-fs.target                               loaded active       active                       Local File Systems
multi-user.target                             loaded active       active                       Multi-User
network.target                                loaded active       active                       Network
remote-fs.target                              loaded active       active                       Remote File Systems
sockets.target                                loaded active       active                       Sockets
swap.target                                   loaded active       active                       Swap
sysinit.target                                loaded active       active                       System Initialization

LOAD   = Reflects whether the unit definition was properly loaded.
ACTIVE = The high-level unit activation state, i.e. generalization of SUB.
SUB    = The low-level unit activation state, values depend on unit type.
JOB    = Pending job for the unit.

221 units listed. Pass --all to see inactive units, too.
[root@lambda] ~#
\end{lstlisting}
\section{References}
\nocite{*}
\bibliography{references}
\bibliographystyle{plain}
\end{document}
