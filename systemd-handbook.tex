\documentclass[titlepage]{article}
\title{Systemd for Administrators}
\usepackage[english]{babel}
\usepackage{hyperref}
\usepackage{listings}
\author{lennart Poettering}
\begin{document}
\maketitle
\tableofcontents
\newpage

\section{Abstract}
As many of you know,
\href{https://www.freedesktop.org/wiki/Software/systemd}{systemd} is the new
Fedora init system, starting with F14, and it is also on its way to being
adopted in a number of other distributions as well (for example,
\href{https://en.opensuse.org/SDB:Systemd}{OpenSUSE}). For administrators
systemd provides a variety of new features and changes and enhances the
administrative process substantially. This blog story is the first part of a
series of articles I plan to post roughly every week for the next months. In
every post I will try to explain one new feature of systemd. Many of these
features are small and simple, so these stories should be interesting to a
broader audience. However, from time to time we'll dive a little bit deeper
into the great new features systemd provides you with.  
\section{References}
\nocite{*}
\bibliography{references}
\bibliographystyle{plain}
\listoffigures
\end{document}
